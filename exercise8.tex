\documentclass[10pt,a4paper]{article}
%\usepackage[ngerman]{babel}
\usepackage{color}
\usepackage{listings}
\usepackage{graphicx}
\usepackage{float}
%\usepackage{moreverb}

\setlength{\parindent}{0pt}

\title{\textbf{Exercise 8 Group 14}}
\author{Jon Volkmar\\
		Heiko Baumann\\
		Peter Hirschfeld}
\date{}
\begin{document}

\maketitle

%\section{}

\begin{enumerate}
\item Please describe in your own words the difference between RDFS amd OWL, and 
then between OWL Lite, OWL DL, OWL Full and OWL 2. Create some simple 
examples to illustrate the difference.

\textbf{Answer:}
OWL is more expressive than RDFS: with OWL we can define relations between classes, constraints and cardinalities, equivalences of classes and properties, create properties of properties, use boolean combinations and other expressive constructs (e.g. disjointnes, inverse properties, property restrictions, enumerations). Also, special properties can be expressed, namely transitive, symmetric, functional, and inverse functional properties. However, some RDF markup is preserved in OWL, and instances of classes are declared in RDF. An example of a transitive property that cannot be expressed in RDFS:

\begin{verbatim}
<owl:ObjectProperty rdf:ID="subRegionOf">
  <rdf:type rdf:resource="&owl;TransitiveProperty"/>
  <rdfs:domain rdf:resource="#Region"/>
  <rdfs:range  rdf:resource="#Region"/>
</owl:ObjectProperty>
\end{verbatim}

OWL Lite is a simplified subset of OWL for tool developers, with some features from OWL, supporting class hierarchies and constraints. However, many features and constructs from OWL are not found in OWL Lite, such as owl:disjointWith.

OWL DL: DL stands for description logic. It is restricted compared to OWL Full in that classes cannot be simultaneously be properties and individuals, and properties cannot simultaneouly be individuals. This is called vocabulary partitioning. As well, the set of object properties and data type properties are disjoint in DL. DL also requires explicit typing. No transitive cardinality constraints can exist in OWL DL. Another difference is that there are some restrictions placed on the usage of anonymous classes.

OWL 2 has several new constructs, such as qualified cardinality restrictions, role chains, and expressive data predicates. It has some different syntaxes, and targets the XML technology toolchain with OWL/XML, which is a new non-RDF XML syntax. It has added features for supporting datatypes, metamodelling, annotation, and database style keys. Property chains are also introduced: for example saying that \textbf{\{ ?x ex:uncle ?y \}} is equivalent to \textbf{\{ ?x ex:parent ?z . ?z ex:brother ?y . \}} can be expressed as a property chain.

\item Ontology modelling application scenario

\begin{enumerate}

\item Prepare a brief project documentation describing your work on the tasks b -– f.
Your documentation should contain the following information:
	\begin{itemize}
		\item Time spent on solving each task (person hours).
			\begin{enumerate}
			\item[(b)] 30 minutes for each person
			\item[(c)] 30 minutes each
			\item[(d)] 30 minutes each
			\item[(e)] 1 hour each
			\item[(f)] 2 hours each
			\end{enumerate}
		\item Time spent discussing each task with your teammate(s).
			\begin{enumerate}
			\item[(b)] 10 minutes
			\item[(c)] 10 minutes
			\item[(d)] 30 minutes
			\item[(e)] 30 minutes
			\item[(f)] 20 minutes
			\end{enumerate}
		\item Difficulties encountered.
			\begin{enumerate}
			\item When making the ontology in Protege, the software would sometimes crash when we tried to run the reasoner. And, if it didn't crash, it would sometimes yield unexpected results. Namely, an individual from class A, defined as disjoint from another class B, would be inferred to be from class B anyway.
			\end{enumerate}
		\item If you encountered difficulties: How did you solve them? How long did it take you
to solve the difficulties (person hours)? Difficulties could be technical issues with
the software used, understandability of the assignment, communication issues
with your teammates, etc...
			\begin{enumerate}
			\item Eventually we abandoned the reasoner in Protege and stopped creating individuals, as everything else appeared to work as expected, and the reasonser is not necessary for the assignment. This took about an hour.
			\item It was difficult to figure out a way to model different class sizes, allowing each class to have a different maximum size but without making a new subclass every time. It was thought about and we settled on a compromise, creating a few subclasses representing different sizes (10, 50, and 100 students) which can be used for any number of courses. This took about half an hour.
			\end{enumerate}
		\item If you needed to apply changes (e.g. while building your ontology you realized
that some competency questions do not work in your scenario, and you needed to
rework your competency questions), document that.
	\end{itemize}

\item Competency questions: Write down at least 20 competency questions which the intended application should be able to answer on the basis of the ontology.

\textbf{Answer:}

\begin{enumerate}

\item For a given course, who is the lecturer or tutor?
\item How many students are enrolled in a given class?
\item Where and when is a lecture given?
\item What are the names of the courses a given student is enrolled in?
\item Which lectures occur on a given day of the week?
\item What are the names of the students enrolled for a given course?
\item How many times does a course occur per week?
\item How many instructors are there for a given course?
\item Who is allowed to loan a given faculty resource (e.g. room, projector, PC...)
\item For a given room, what dates is it booked for?
\item What is the maximum loan period for a given faculty resource?
\item Is a resource currently on loan?
\item What is the capacity of a given course?
\item Is a given course a lecture or a seminar?
\item What prior courses are required for a student to participate in a given courses?
\item Which courses is a student eligible to enroll for?
\item For a given student, which courses need to be completed before they can enroll in a certain course?
\item Which resources is a given person allowed to loan?
\item Is a particular resource loanable?
\item What are the names of all the students enrolled in a course which takes place on a Monday?
\end{enumerate}

\item Glossary: Extract key terms from the competency questions which frequently occur and are of importance for the given application domain. 

\textbf{Answer:}

\begin{itemize}
\item Course
\item Lecturer
\item Tutor
\item Student
\item Course Name
\item Course capacity
\item Loan period
\item Booked dates
\item Lecture 
\item Seminar
\item Required courses
\item Loanability
\end{itemize}

\item Mindmap: Create mindmaps to model the relationships between the key terms of your glossary in a light-weight way.

\textbf{Answer:}

Mindmap created with Freemind:

\begin{figure}[H]
  \caption{Mindmap of the "Faculty" ontology}
  \centering
    \includegraphics[scale=0.7]{Faculty_mindmap.png}
\end{figure}

\item Ontology engineering: Derive from the mindmaps an ontology containing some first classes. Use a ontology editor for this task, e.g., Protégé(http://protege.stanford.edu/). Add labels in English and German language to each concept of your ontology

\textbf{Answer:}

Ontology created with Protege. Please see ex8.owl for the OWL/XML markup. The following diagram, generated by the OwlViz tool in Protege, shows the classes of the ontology. It does not show the object and data properties which are used in restrictions fufilling our scenario requirements, to view these the owl file must be opened in Protege or simply in a text editor.

\begin{figure}[H]
  \caption{Classes in the OWL representation of the "Faculty" ontology}
  \centering
    \includegraphics[scale=0.7]{owl_graph.png}
\end{figure}

\item Go to the Watson Website (http://watson.kmi.open.ac.uk/WatsonWUI/) and
search for existing ontologies modeling the same or similar concepts as you have
in your ontology. Refine your ontology by reusing concepts and relationships of
the found ontologies.
\end{enumerate}

We were not able to find any existing ontology close enough to our own to be easily adaptable. However, maybe it would be possible to model "temporal events" such as meetings and generic time intervals, as done in (http://daml.umbc.edu/ontologies/cobra/0.4/academia).
\end{enumerate}


\end{document}